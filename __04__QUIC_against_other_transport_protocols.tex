\section{QUIC overview}
QUIC is a connection-based, multistream protocol that uses UDP under the hood. 
It is integrated with TLS and provides own congestion control and flow control mechanisms.
Data sent over QUIC is delivered in a reliable, ordered way.
Together with QUIC there comes a new version of HTTP called HTTP/3.

This section outlines main features of QUIC and compares QUIC to the other transport protocols.

\subsection{Security}
QUIC is integrated with TLS 1.3 providing confidentiality, integrity and authenticity of its packets.
Integration with TLS 1.3 is described in details in RFC 9001 \cite{rfc9001} and is based on negotiating both transport and cryptographic parameters in a single connection handshake.
This results in low-latency connection establishment (see section \ref{subsec:low-latency-conn-est}).
On the other hand there is no possibility to send unencrypted data over QUIC.

\subsection{Low-latency connection establishment}
\label{subsec:low-latency-conn-est}
Integration with TLS 1.3 results in reduction of handshakes needed for establishing a connection.
Unlike TCP, QUIC can perform both transport and cryptographic handshakes simultaneously by sending in its Initial packet data required by TLS.
This is presented in figure \ref{fig:low-latency-conn-est}.

\begin{figure}
    \centering
    \begin{subfigure}{.5\textwidth}
        \begin{sequencediagram}
            \newinst{client}{Client}
            \newinst[3]{server}{Server}
            \mess{client}{TCP SYN}{server}
            \mess{server}{TCP SYN + ACK}{client}
            \mess{client}{TCP ACK}{server}
            \postlevel
            \mess{client}{TLS ClientHello}{server}
            \mess{server}{TLS ServerHello}{client}
            \mess{client}{TLS Finished}{server}
            \postlevel
            \mess{client}{HTTP REQ}{server}
            \mess{server}{HTTP RES}{client}
        \end{sequencediagram}
        \caption{HTTP request in TCP}
        \label{subfig:http-req-tcp}
    \end{subfigure}%
    \begin{subfigure}{.5\textwidth}
        \begin{sequencediagram}
            \newinst{client}{Client}
            \newinst[3]{server}{Server}
            \mess{client}{QUIC}{server}
            \mess{server}{QUIC}{client}
            \mess{client}{QUIC}{server}
            \postlevel
            \mess{client}{HTTP REQ}{server}
            \mess{server}{HTTP RES}{client}
        \end{sequencediagram}
        \caption{HTTP request in QUIC}
        \label{subfig:http-req-quic}
    \end{subfigure}
    \caption{HTTP request comparison}
    \label{fig:low-latency-conn-est}
\end{figure}

Standard HTTP request using TCP/TLS stack requires 3-RTT (Round Trip Time).
Round Trip Time is a time needed for transporting a message from one side to the other and back again.
In TCP/TLS scenario, we need to perform separate handshakes for TCP and TLS which results in 2-RTT.
After this time we are able to send HTTP request which takes additional 1-RTT.
This scenario is presented in figure \ref{subfig:http-req-tcp}.

QUIC takes another approach.
It combines transport and cryptographic handshakes reducing time needed for establishing a connection.
In such a case sending HTTP request and receiving a response takes 2-RTT. 
This scenario is presented in figure \ref{subfig:http-req-quic}.

However, QUIC can reduce time needed for sending HTTP request and receiving a response even more.
In case when there was communication with the server previously and client cached information from it, it is possible to send HTTP request in the first round-trip of the connection.
This scenario is presented in figure \ref{fig:http-req-quic-0rtt}
\begin{figure}
\centering
    \begin{sequencediagram}
        \newinst{client}{Client}
        \newinst[3]{server}{Server}
        \mess{client}{QUIC}{server}
        \mess{client}{HTTP REQ}{server}
        \postlevel
        \mess{server}{QUIC}{client}
        \mess{server}{HTTP RES}{client}
        \postlevel
        \mess{client}{QUIC}{server}
    \end{sequencediagram}
\caption{HTTP request in QUIC with 0-RTT packets}
\label{fig:http-req-quic-0rtt}
\end{figure}


\subsection{Stream multiplexing}
QUIC is a multiplexed protocol which means we can open many streams in one QUIC connection.
Stream is an ordered sequence of bytes.
Each stream can be unidirectional or bidirectional and can be opened by a client or by a server.
Different types of streams can have different flow control limits (see section \ref{subsec:flow-control-and-congestion-control}).
Data in specific streams is carried independently which means that losing packet in one stream does not affect flow in other streams.
Figure \ref{fig:stream-multiplexing} shows this scenario.
One of the packets in stream 1 (marked as gray) was lost and needs to be retransmitted blocking subsequent packets (marked as black) on the receiver side.
They cannot be conveyed to the application layer until retransmission and reordering of the lost packet.
This is the source of so called head of line blocking problem which appears when we are trying to multiplex many HTTP requests in a single TCP connection.
Losing one packet stops the entire flow.
However, thanks to the stream multiplexing QUIC resolves this problem and streams 2 and 3 are not affected by the packet loss in stream 1.

\begin{figure}[h]
\centering
\begin{tikzpicture}

\node[cylinder, 
    draw = violet, 
    text = purple,
    style={transform shape},
    cylinder uses custom fill, 
    cylinder body fill = magenta!10, 
    cylinder end fill = magenta!40,
    minimum size = 3.5cm] (c) at (0,0) {QUIC connection};
    
\draw [->, postaction={decorate,decoration={raise=2ex, text along path,text align=center,text={stream 1}}}] (2, 1.5) -- (7.5, 1.5);
\filldraw [fill=green!20, draw=black] (2.5, 1.25) rectangle (3.25, 1.75);
\filldraw [fill=gray!20, draw=black] (3.75, 1.25) rectangle (4.5, 1.75);
\filldraw [fill=black!60, draw=black] (5, 1.25) rectangle (5.75, 1.75);
\filldraw [fill=black!60, draw=black] (6.25, 1.25) rectangle (7, 1.75);

\draw [->, postaction={decorate,decoration={raise=2ex, text along path,text align=center,text={stream 2}}}] (2, 0) -- (7.5, 0);
\filldraw [fill=cyan!20, draw=black] (2.5, -0.25) rectangle (3.25, 0.25);
\filldraw [fill=cyan!20, draw=black] (3.75, -0.25) rectangle (4.5, 0.25);
\filldraw [fill=cyan!20, draw=black] (5, -0.25) rectangle (5.75, 0.25);
\filldraw [fill=cyan!20, draw=black] (6.25, -0.25) rectangle (7, 0.25);

\draw [->, postaction={decorate,decoration={raise=2ex, text along path,text align=center,text={stream 3}}}] (2, -1.5) -- (7.5, -1.5);
\filldraw [fill=purple!20, draw=black] (2.5, -1.75) rectangle (3.25, -1.25);
\filldraw [fill=purple!20, draw=black] (3.75, -1.75) rectangle (4.5, -1.25);
\filldraw [fill=purple!20, draw=black] (5, -1.75) rectangle (5.75, -1.25);
\filldraw [fill=purple!20, draw=black] (6.25, -1.75) rectangle (7, -1.25);


\end{tikzpicture}
\caption{Stream multiplexing in a single QUIC connection}
\label{fig:stream-multiplexing}
\end{figure}

\subsection{Connection migration}
Each QUIC connection has a set of connection identifiers each of which can identify the connection.
Connection IDs can be variable length.
QUIC uses connection IDs to route packets to proper endpoint.
This way, as long as there is an alternative network path, QUIC is resilient to network changes avoiding restarting the connection when an endpoint changes its transport address (IP or port).

\subsection{Flow control and congestion control mechanisms}
\label{subsec:flow-control-and-congestion-control}

\subsection{Reliability}
QUIC is a reliable protocol which means it guarantees delivery of each packet sent by the endpoint.
Packets are delivered to the receiver side in the order they were sent.
However, there are different IETF drafts that expand QUIC with new features.
One of them called "An Unreliable Datagram Extension to QUIC"\cite{bider-ssh-quic-09} allows for sending unreliable messages over QUIC.
This can take place in simultaneously to the reliable communication.
Unreliable messages are not subject to flow control mechanisms however, they are congestion controlled.
Each unreliable message is also acknowledged so that application layer can be provided with the packet loss information.
Section \ref{sec:datagrams} describes Datagram extension in details.

\subsection{QUIC against other transport protocols}
\label{sec:quic_against_other_transport_protocols}
This section compares QUIC with three main transport protocols -- TCP, UDP and SCTP.  
\subsubsection{TCP}
Transmission Control Protocol -- connection based, stream oriented and reliable transport protocol.
It guarantees the order of messages and provides flow control and congestion control mechanisms.
One of the drawbacks of TCP is its head of line blocking problem.
If one TCP segment gets lost then all subsequent TCP segments that arrived to the recipient have to be buffered until retransmission of the lost segment.
When the lost segment finally arrives, buffered segments and retranssmited segment are arranged in the proper order and made available to the user who can read them from internal socket buffer.
This is the reason why HTTP/2 only partially solves head of line blocking problem of HTTP/1 -- there is still the same problem on TCP level.
\subsubsection{UDP}
User Datagram Protocol -- connection less, unreliable transport protocol. 
It does not preserve the order of messages. 
Datagrams, in this protocol, are not acknowledged nor congestion controlled.
It does not also introduce any flow control mechanisms.
Thanks to its simplicity and low bandwidth affection it is the base for RTP protocol which is widely used for transmitting multimedia.
Unlike TCP, UDP also allows for multicast transmission.
However, multicast is rarely used, mostly in local networks.
\subsubsection{SCTP}
Stream Control Transmission Protocol -- reliable, based on UDP, message oriented transport protocol.
Unlike TCP, SCTP is multi-stream protocol which means that it is not affected by head of line blocking problem. 
SCTP is partially ordered but it also allows for sending unordered messages. 
Another feature of SCTP is multi-homing in which an endpoint uses two different addresses. 
This introduces some kind of fault tolerance -- one address is designated as a primary and the other one can be used in case of failure of the first one.
Unfortunately SCTP is not widely used.
Its main domain is telecommunication as a lot of home routers does not handle SCTP properly.

\subsubsection{Summary}
Table \ref{tab:protocols_comparision} presents a short summary of transport protocols comparison.
\begin{table}[h]
\centering
\begin{tabular}{|c | c | c | c | c |} 
    \hline
    & TCP & UDP & SCTP & QUIC \\  
    \hline
    reliability & reliable & unreliable & reliable & reliable \\
    \hline
    transmission & byte oriented & message oriented & message oriented & byte oriented \\
    \hline
    flow control & yes & no & yes & yes \\
    \hline
    congestion control & yes & no & yes & yes \\
    \hline
    packets order & ordered & unordered & partially ordered & ordered \\
    \hline
    multistream & no & no & yes & yes \\
    \hline
\end{tabular}
\caption{\label{tab:protocols_comparision}Transport protocols comparison.}
\end{table}
