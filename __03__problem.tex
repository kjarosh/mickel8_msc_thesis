\section{Problem}
\label{sec:problem}
% Sformułowanie  problemu.Zawiera  precyzyjne,  wyczerpujące  definicje problemów badawczego  i  projektowego,  jakie  zostaną w  pracy  rozwiązane. Sformułowania  te wykorzystują aktualny  formalizm  oraz  terminologię  specyficzne  dla  dziedziny pracy. Są  one odniesione do  aktualnego  stanu  wiedzy,  co  potwierdzać  muszą  stosowne referencje. Ta część pracy zawiera w szczególności analizę zapotrzebowań projektu software  będącego pożądanym składnikiem  pracy magisterskiej w    naukach technicznych w dyscyplinie  informatyka.
This section outlines problems that will be addressed in this document.

\subsection{Complexity of current systems}
One of 

\subsection{Thesis}
The objective of this document is to overview QUIC in terms of interactive communication and answer the question if QUIC can improve interactive communication.
Improvement means:
\begin{itemize}
    \item reducing delay -- it is important in terms of Tactile Internet where delay of 1ms is desirable
    \item reducing bandwidth usage -- significant for poor connections or networks
    \item better adoption to changing network parameters and faster recovery after packet loss
    \item reducing complexity of current systems -- the simpler system is, the easier it is to monitor, maintain and develop
\end{itemize}
